% !TeX root = report_example.tex
\newcommand*{\PathToAssets}{../assets}%
\newcommand*{\PathToOutput}{../_output}%
% \newcommand*{\PathToBibFile}{bibliography.bib}%


%%%%%%%%%%%%%%%%%%%%%%%%%%%%%%%%%%%%%%
%% This file is compiled with XeLaTex.
%%%%%%%%%%%%%%%%%%%%%%%%%%%%%%%%%%%%%%
\documentclass[12pt]{article}
%\documentclass[reqno]{amsart}
%\documentclass[titlepage]{amsart}
\usepackage{my_article_header}
\usepackage{my_common_header}

\begin{document}
\title{
Corporate CDS-Bond Spread
}
% {\color{blue} \large Preliminary. Do not distribute.}
% }

\author{
Alex Wang \& Vincent Xu
  % \newline 
  % \hspace{3pt} \indent \hspace{3pt}
  % % I am immensely grateful to...
}
% \maketitle
\begin{titlepage}
% \input{cover_letter.tex}
\maketitle
%http://tex.stackexchange.com/questions/141446/problem-of-duplicate-identifier-when-using-bibentry-and-hyperref
% \nobibliography*

% Abstract should have fewer than 150 words

\doublespacing
\begin{abstract}
We attempt to replicate the Corporate Bond - Credit Default Swap arbitrage findings from 
Siriwardane, Sunderam, and Wallen 2023. While the exact results do not match up,
the process of pulling and analyzing the data was constructive and informative
regarding using data from the Wharton Research Data Services (WRDS) platform.

\end{abstract}


\end{titlepage}

\doublespacing
\section{Introduction}

We attempt to replicate the findings of Siriwardane, Sunderam, and Wallen from 2023
on the CDS-Corporate Bond spread. We encountered numerous roadblocks during the
data pulling process due to insufficient description of methodology from the
original authors. Nevertheless, we followed their steps as best as we could, and
ultimately constructed a graph of the spread according to data pulled and analyzed
using our own methodology. We successfully identified a reasonable arbitrage
opportunity as seen in the figure below. We do not, however, comment on the exact 
feasibility of the trade strategy as our analysis did not incorporate transaction
costs and required a degree of creative freedom in the process.

\begin{figure}
    \centering
    \caption{Plot of CDS-Corporate Bond Spread}
      \centering
      \includegraphics[width=0.75\linewidth]{\PathToOutput/Time Series Plot of rfr.png}
    
    \caption*{
      Implied arbitrage opportunity (measured in basis points) between a given 
      corporate bond and its corresponding CDS' par-rate, after accounting for the 
      contemporary treasury yield
      }
    \end{figure}

\section{Data pull and analysis methodology}

We used data obtained from various tables on WRDS.
For Bonds, we used the Mergent fixed income\footnote{fisd\_fisd.fisd\_mergedissue} and 
WRDS bond return\footnote{wrdsapps\_bondret.bondret} tables.
For Credit Default Swaps we used the Markit CDS\footnote{markit.CDS}
tables across the years of 2002-2024. To match the data between CDSs and Bonds, we
used the Markit RedCode CUSIP table\footnote{markit\_red.redobllookup} to match 
CDS RedCodes to Bond CUSIPs.

First we pull corporate bond time series data, treasury time series data, 
and some mapping features. The corporate bond time series data was used to 
generate the FR value (calculated as the Z-spread of the corporate bonds). 
Treasury time series data was also used for this purpose as we generated treasury yields.

The Z-Spread of a Corporate Bond is the constant spread that must be added to 
the zero-coupon Treasury yield curve to make the present value of the bond's 
cash flows equal to its market price. 

The Z-spread is found by solving the equation:
\begin{equation}
    P = \sum_{t=1}^{N} \frac{C_t}{(1 + y_m + z)^t}
\end{equation}


Where: \\
- $P$ = Market price of the bond \\
- $C_t$ = Cash flow (coupon or principal) at time $t$ \\
- $r_t$ = Discount rate of the zero-coupon Treasury bond at time $t$ \\
- $Z$ = Z-spread \\
- $N$ = Number of periods \\

The Z-spread accounts for different discount rates of a treasury. 
Our surrogate Z-spread (since a Z-spread was not provided in the WRDS database)
and in turn what we used to calculate FR, is as follows:

\begin{equation}
P = \sum_{t=1}^{N} \frac{C_t}{(1 + y_m + z)^t}
\end{equation}

Where: \\
- $P$ = Market price of the bond \\
- $C_t$ = Cash flow (coupon or principal) at time \( t \)  \\
- $y_m$ = Yield of the zero-coupon Treasury bond at time $t_m$ \\
\indent - This is different because the yield is calculated at the time the 
market price is set, which is a fixed rate substitute for a varying discount rate
- $z$ = Z-spread surrogate \\
- $N$ = Number of periods \\

We continue a more in depth discussion regarding our data processing methodology
in the available Jupyter notebook.

\newpage
\bibliographystyle{jpe}
\bibliography{bibliography.bib}  % list here all the bibliographies that you need. 
Siriwardane, Emil and Sunderam, Aditya and Wallen, Jonathan, Segmented Arbitrage 
(November 2023). Available at SSRN: https://ssrn.com/abstract=3960980 or 
http://dx.doi.org/10.2139/ssrn.3960980


\end{document}
